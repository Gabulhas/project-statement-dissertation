\chapter*{Resumo}

Esta dissertação aborda os desafios complexos e soluções associados à implementação e teste de algoritmos de consenso em redes \textit{blockchain}. A pesquisa é baseada na plataforma \textit{blockchain} \textit{Tezos}, escolhida pela sua adaptabilidade e versatilidade. O objetivo principal é avançar no campo da tecnologia \textit{blockchain}, desenvolvendo ferramentas e métodos que facilitem a troca fácil, teste e análise comparativa de diferentes algoritmos de consenso em um ambiente de rede ao vivo.

O documento está organizado em várias seções cruciais. Começa com uma introdução que delineia o problema e a solução, enfatizando as lacunas existentes no desenvolvimento e teste de algoritmos de consenso, particularmente a ausência de ferramentas para testes ao vivo e os desafios na troca de algoritmos. 
Segue-se uma revisão abrangente do estado da arte em algoritmos de consenso, oferecendo uma análise detalhada da sua eficiência, limitações e adequação para várias aplicações.

Um estudo de caso preliminar é então apresentado, focando na implementação de um algoritmo de consenso de Prova de Trabalho (PoW na plataforma Tezos. Isso serve como uma prova de conceito e fornece insights valiosos sobre o funcionamento interno dos algoritmos de consenso e sua integração com nós de \textit{blockchain} existentes.

A contribuição central desta dissertação é o desenvolvimento de duas ferramentas: uma voltada para facilitar o desenvolvimento de algoritmos de consenso e outra projetada para seus testes ao vivo. Essas ferramentas destinam-se a servir tanto a comunidade acadêmica quanto a industrial interessadas em tecnologia \textit{blockchain}.

Por fim, a dissertação apresenta um cenário do mundo real que emprega ambas as ferramentas. Um algoritmo de Prova de Autoridade (PoA) é implementado usando a ferramenta de desenvolvimento e depois testado  fazendo uso da ferramenta de teste ao vivo. Os resultados são comparados com o algoritmo PoW previamente implementado, fornecendo evidências empíricas da eficiência e limitações de ambos os algoritmos.

Em resumo, esta dissertação oferece um \textit{framework} robusto para o desenvolvimento e teste de algoritmos de consenso, substanciado por evidências empíricas da plataforma Tezos. Tem implicações significativas para ambos os pesquisadores acadêmicos e \textit{stakeholders} da indústria, estabelecendo uma base sólida para trabalhos futuros neste campo em rápida evolução.
