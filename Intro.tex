\chapter{Introduction}
\label{chap:int}


\section{Introduction}

Blockchains are a revolutionary technology that have the potential to transform various industries by providing a secure and transparent platform for storing and exchanging information and assets. Decentralized in nature, blockchains use cryptography and consensus algorithms to create a shared ledger system, eliminating the need for a trusted intermediary.

In recent years, the demand for secure and transparent systems has led to the growing popularity of blockchains, especially in finance, supply chain management, digital identity, and digital assets. The decentralization of blockchains reduces the risk of single points of failure and ensures the integrity of data stored on the network.

Consensus algorithms play a critical role in the functioning of blockchains, determining the process of validating transactions, adding new blocks to the chain, and other crucial aspects like the selection of the next block producer. Different consensus algorithms offer varying levels of security, scalability, and decentralization, making the choice of the right algorithm for a specific use case crucial for the success of a blockchain network.


The main objective of this dissertation is to develop tools and methods to easily swap, test, develop, to compare different consensus algorithms in a live environment using actual nodes on a network. This study aims to provide insight into the field of blockchains and consensus algorithms, contributing to the advancement of more secure, scalable, and efficient blockchain systems. The Tezos \cite{goodman2014tezos} blockchain node will serve as a testbed for this project, and the focus will be on examining the potential of consensus algorithms to be swapped and changed on demand.

This dissertation will provide valuable insights into the field of blockchains and consensus algorithms, contributing to the development of more secure, scalable, and efficient blockchain systems. From tools and methods to test consensus algorithms to the development of algorithms and their integration into existing blockchain nodes, this dissertation will leave a lasting impact on the future of blockchains.

\section{Motivation}

% Despite the growing popularity of blockchains, there is currently a lack of tools and methods for easily swapping and testing different consensus algorithms in a live environment.
% This makes it difficult to determine which algorithm is best suited for a specific use case, particularly in the context of the blockchain trilemma, where it is challenging to achieve optimal balance between scalability, security, and decentralization.
% 
% Additionally, developing consensus algorithms can be a complex and time-consuming task, and there is currently no standardized method for describing and implementing them.
% This presents a significant barrier to the adoption and evolution of blockchains, as it limits the ability of developers to experiment with new and innovative consensus algorithms. 
% Also, the decentralization being one of the strenghts points of blockchain networks falls in the hands of a few selected developers.
% 
% This dissertation aims to address these challenges by providing a comprehensive analysis of different consensus algorithms for blockchains, and exploring the potential for consensus algorithms to be plugged and changed on demand.
% The results of this study will serve as valuable feedback for the development of a domain-specific language for describing consensus algorithms, and will provide insights into the most effective methods for testing consensus algorithms in a live environment.
% 
% The importance of this research cannot be overstated, as the ability to easily swap and test consensus algorithms is critical for the continued growth and evolution of blockchains.
% This will not only benefit the academic community but also industry stakeholders, who will be able to make informed decisions about which consensus algorithm is best suited for their specific use case.
% 
% In conclusion, this dissertation is motivated by the growing need for a comprehensive analysis of consensus algorithms for blockchains, and the desire to explore new and innovative methods for swapping testing protocols in a non-simulated way.
% The objective of this study is to contribute to the development of a standardized method for describing and testing consensus algorithms, ultimately advancing the field of blockchains and improving the security, scalability, and decentralization of blockchain networks.
% 

Blockchain technology has rapidly gained popularity in recent years, yet the development and implementation of consensus algorithms for blockchains remains a complex and time-consuming task. The lack of tools and methods for easily swapping and testing different consensus algorithms in a live environment presents a significant barrier to the adoption and evolution of blockchains, limiting the ability of developers to experiment with new and innovative consensus algorithms. This is particularly concerning given the importance of consensus algorithms in achieving optimal balance between scalability, security, and decentralization, also known as the ``blockchain trilemma''.

This document aims to address these challenges by exploring the potential for consensus algorithms to be plugged and changed on demand. The study will provide valuable insights into the most effective methods for testing consensus algorithms in a live environment, and contribute to the development of a domain-specific language for describing consensus algorithms. The results of this research will benefit both the academic community and industry stakeholders, who will be able to make informed decisions about which consensus algorithm is best suited for their specific use case.

The objective of this study is to advance the field of blockchains by improving the security, scalability, and decentralization of blockchain networks. The importance of this research cannot be overstated, as the ability to easily swap and test consensus algorithms is critical for the continued growth and evolution of blockchains.

In conclusion, this project is motivated by the growing need for a comprehensive analysis of consensus algorithms for blockchains and the desire to explore new and innovative methods for swapping and testing protocols in a live environment. The goal is to contribute to the development of a standardized method for describing and testing consensus algorithms, ultimately reducing barriers to the adoption and evolution of blockchains.

\section{Document Organization}

This document is organized as follow:
\begin{enumerate}
    \item \textbf{Core Concepts and State of the Art} - This chapter aims to provide an overview of the Core Concepts necessary to understand the inner workings of a Blockchain network, as well as to explore the most commonly used consensus algorithms and a Domain-Specific Language used that describes this type of algorithms.
    \item \textbf{Problem Statement} - In this chapter we will describe the problem we are solving and a solution proposed to solve it. In it we will also elaborate on a experiment done to further sustain that the solution is a feasiable one and, to finish the chapter, the tasks for the continuation of the development of this dissertation will be uncovered.
\end{enumerate}

This is the organization of the document, where firstly we describe how this type of system works, how they can differ and then we elaborate on the problem and solution that will be worked on in this dissertation.
