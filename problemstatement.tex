\chapter{Problem Statement, Experiments and Work Plan}

The Problem Statement, Experiments, and Work Plan chapter is a crucial component of the thesis as it outlines the research problem, the proposed solution, and the plan for executing the research. In this chapter, we aim to present a clear and concise description of the problem we are solving, why it is important, and why the proposed solution is a valid one. The chapter will also include a detailed description of the experiments that will be conducted to validate the solution, as well as the work plan that outlines the tasks and the timeline for executing the research.

\section{The Problem}
The development of blockchain technology has led to the creation of a large number of consensus protocols, each with its unique characteristics, from the selection of the leader, to the type of network, data structure, block structure, time between each block, and multiple ways to have a chain, to name just a few. Newer and innovative protocols are designed ``everyday'' with totally different characteristics compared to previous protocols.
This abundance of choices has made it difficult for individuals or organizations looking to start their own blockchain to determine which protocol is the best fit for their needs, as there's a lot to account for.

Furthermore, there is no easy way to test and compare these consensus protocols in a live, non-simulated environment.
There is no blockchain node software that allow for a seamless swap of the consensus algorithm, making it challenging to properly evaluate the performance of each protocol. This presents a major obstacle for researchers and developers looking to experiment with and improve upon existing consensus protocols.

Additionally, the process of designing and coding a new consensus algorithm is also challenging and requires a significant amount of expertise and resources. There is currently no platform that allows for the easy design, swap, and testing of new consensus algorithms.

All these problems contribute to the lack of standardization in the field of consensus protocols, a lot of attributes and characteristic to take in, making it difficult for individuals and organizations to adopt blockchain technology effectively.
It is crucial to find a solution that allows for the easy and efficient comparison and testing of consensus protocols, as well as the design and implementation of new protocols. This will pave the way for the further development and widespread adoption of blockchain technology.


\section{Solution Proposed}
The proposed solution aims to tackle the challenges mentioned in the previous section by providing a method of easily swapping consensus algorithms, so they can then be tested. To achieve this, the solution leverages the advantages of a pre-existing blockchain that is suitable for both swapping and testing.

By using an already built blockchain, the focus of this solution is solely on the consensus protocol, the crucial aspect of any blockchain. The need for implementing other parts of the blockchain node such as the Peer-to-Peer layer, client, validation layer, storage of the blockchain state, etc., is eliminated. This benefits the solution by making use of already tested and industrial-grade blockchain software, freeing up time and resources to concentrate solely on the consensus.

One of the ideas for implementing this solution is to use the Tezos blockchain. The reason for using Tezos is its inherent adaptability, which allows for the removal of its current consensus algorithm and the implementation of a new, customizable one. This approach provides a layer of adaptability to the Tezos blockchain, making it suitable for testing and swapping various consensus algorithms with ease.

\textbf{As of the writing of this document, there's no knowledge of projects or documents that describe a similar solution to this one proposed}. There exists the concept of consensus testing when talking about a blockchain node implementation in specific, that is, for example, tools to test the Ethereum network, or the Tezos network, but there are no tools to test specifically the consensus part of this projects and compare them.

In conclusion, the solution proposes to overcome the challenges mentioned in the previous section by providing a method of testing and easily swapping consensus algorithms. By leveraging the advantages of a pre-existing blockchain and its adaptability, the focus remains solely on the consensus algorithm, the crucial aspect of any blockchain, allowing for efficient and effective testing and comparison of different consensus algorithms.


\section{The Experiment}

The Tezos blockchain has proven itself to be a flexible and adaptable platform, capable of accommodating a wide range of applications and use cases. This makes it a suitable choice for an experiment that aims to test and demonstrate the implementation of a consensus protocol. In this section, we will elaborate on why Tezos is an ideal fit for this experiment and explore the inner workings of the platform that make it so.

The experiment at hand involves the implementation of a Proof of Work consensus protocol, similar to the one used by the Bitcoin blockchain.
As an exercise, this protocol was implemented on the Tezos blockchain and its workings will be discussed in detail. This will include a discussion of the requirements for a protocol, its integration with the Tezos node, the implementation of the protocol itself and the tools used to interact with it.

We will also delve into the implementation of a client and a baker, which are integral components of the protocol. The client is used to access information within the network, while the baker is responsible for the creation and mining of blocks. This includes the process of mining the block by finding the nonce that makes the block's hash lower than the target, which is a key point in the execution of a Nakamoto like consensus.

In conclusion, this experiment will serve as a demonstration of the flexibility and adaptability of the Tezos blockchain and its ability to accommodate different consensus protocols. The implementation of the Proof of Work consensus protocol will serve as a proof of concept and highlights the potential of the platform to support a wider range of consensus protocols, even those different from the orginal protocol implemented.

\subsection*{\textbf{Tezos Blockchain as a tool for the Experiment}}
The choice of Tezos as the blockchain network for this experiment is a deliberate one, as it aligns well with the goals of the solution presented in the previous section.

To understand why Tezos was chosen, it is important to first explain what Tezos is. Tezos is a blockchain network that was launched in 2018. The history of Tezos began with the idea of creating a generic and self-amending crypto-ledger, which could be improved and upgraded over time without causing any disruption to its community. This makes Tezos stand out from other blockchain networks, which often struggle with the challenges of upgrading their underlying technology. The idea behind Tezos is to provide a blockchain network that is not only secure, but also flexible and upgradable. The network is designed to be self-amending, meaning that changes to its protocol can be made without the need for a hard fork. Tezos originally operates on a Proof of Stake consensus, but the specifics of this consensus are irrelevant since we are taking it out and swapping with different consensus protocols.
% TODO: explain what hard fork is


When compared to other blockchain networks, Tezos stands out as a really good fit for this experiment. It is commercially and industrially used and its upgradability makes it suitable for the replacement of the consensus algorithm. Additionally, the protocol in the codebase is independent and the whole project is designed for the replacement of such. 
The idea of the protocol in Tezos is stateless, meaning that it is independent from other parts of the node like the Peer-to-Peer layer, the client, the validation layer, and the storage of the blockchain state on disk.
This makes it easy to focus on writing the actual protocol and eliminates the need to worry about other components.

In addition to its technological strengths, Tezos is also a well-tested and industry-grade blockchain network, used by millions of users. This makes it a good fit for the experiment described in this dissertation, as it can be relied upon to provide a stable and secure platform for testing new consensus algorithms.


In conclusion, using Tezos for this experiment was a strategic decision as it offers the necessary features and characteristics to carry out the experiment successfully. Its history, design, and upgradability make it an ideal candidate for this experiment, and the stateless nature of the protocol part of the codebase allows the focus to be on the consensus, rather than other components of the node.
The results of this experiment will provide valuable insights into the potential of Tezos and its adaptability, making it an exciting step in the development of a blockchain network that can easily swap consensus algorithms.



\subsection*{How it's structured and How it allows self-amending}

Tezos has a unique structure, which separates the protocol, or how the Tezos project calls it, ``Economic Protocol'' from the rest of the node, known as the shell. 
The protocol is responsible for interpreting transactions and administrative operations, as well as detecting erroneous blocks.

The shell includes the validator, which selects the valid head with the highest absolute score, the peer-to-peer layer, the disk storage of blocks, and the versioned state of the ledger. The distributed database abstracts the fetching and replication of new chain data to the validator.

The economic protocol on the chain is subject to an amendment procedure, which allows for the on-chain operations to switch from one protocol to another.

Finally, the RPC (Remote Procedure Call) layer is an important part of the Tezos node, allowing clients, third-party applications, and daemons to interact with the node and introspect its state using the JSON format and HTTP protocol. This component is fully interoperable and auto-descriptive, using JSON schema.

In other words, the outside world of the node communicates with the shell, via RPC layer calls, and the shell communicates with the protocol. Like said previously, the protocol is stateless, and is only used to preform the logic part of the consensus, that is, to verify transactions, blocks and to describe what the shell should do when receiving newer information.

The communication between the shell and the protocol in Tezos is based on an interface called ``Updater.PROTOCOL''.
The protocol component is restricted to a specific environment that restricts the access to a defined set of OCaml modules. This is to improve the security of the protocol.
The protocol must implement the Updater.PROTOCOL interface (defined in the environment) in order to interact with the shell. The interface requires the protocol to define protocol-specific types for operations/transactions and the block header, along with encoders/decoders for the types.

The protocol must also provide functions for processing blocks and updating the context, which represents the protocol state. The context is stored as a disk-based immutable key-value store. The block header and operations also have a shell header (protocol-independent) and a protocol-specific header.

A Tezos node can contain multiple economic protocols, but only one of them is activated at any given time, and the it always starts with the ``genesis'' protocol.
The protocols are linked to the node at the time of compilation, and some protocols can also be registered dynamically at runtime via an RPC.

Protocol activation in Tezos is a two-step process. Firstly, a command injects an ``activation block'' to the blockchain. This block contains only one operation, which is to activate the next protocol. The activation block is the only block using the genesis protocol, as this protocol doesn't contain any other functionality besides the activation of a different protocol.
Secondly, the next block in the blockchain will be the first block using the activated protocol. The activation command requires the hash of the protocol to be activated, and the protocol must be registered.

In conclusion, the structure of Tezos is unique in that it separates the protocol, also known as the ``Economic Protocol'', from the rest of the node, called the shell. The protocol is responsible for interpreting transactions and administrative operations, while the shell includes the validator, peer-to-peer layer, disk storage of blocks, and versioned state of the ledger. The economic protocol on the chain is subject to amendment procedures, allowing for on-chain operations to switch from one protocol to another. The RPC layer allows clients, third-party applications, and daemons to interact with the node and its state. The communication between the shell and the protocol is based on the Updater.PROTOCOL interface, and the protocol is restricted to a specific environment to improve security. A Tezos node can contain multiple economic protocols, but only one of them is activated at any given time through a two-step activation process.


\subsection*{\textbf{Implementation of a Protocol}}

Implementing a Proof of Work consensus algorithm as a protocol for Tezos was motivated by several reasons. Firstly, it was an opportunity for hands-on learning about how protocols are implemented and structured. This understanding is crucial for later work, such as testing and adding an adaptive layer to the DSL mentioned in previous sections. It also allowed for a deeper understanding of how Tezos handles protocols and how they are executed in the Tezos node.

The implementation of Proof of Work was also significant because it provides a contrast to Tezos' original Proof of Stake protocol. This demonstrated that Tezos can be used to implement other types of consensus algorithms, including those that are different from the original. The implementation of Proof of Work also has the whole the concept of mining, which is not present in the Proof of Stake protocol.

Additionally, implementing a protocol in Tezos is a big accomplishment. The fact that only a few people do so makes this achievement even more significant, especially considering that it's a different type of consensus. This demonstrates a deep understanding of consensus algorithms and the ability to put that knowledge into practice.

In conclusion, the implementation of Proof of Work as a protocol for Tezos was a valuable learning experience that allowed for a deeper understanding of how protocols are structured and executed in Tezos. It also reinforced the concept of consensus algorithms and demonstrated the ability to put that knowledge into practice by implementing a unique protocol in Tezos.


\subsection*{Requirements of a Protocol In Tezos}
In order to implement the experiment of a Proof of Work consensus algorithm for Tezos, a "lib\_protocol" module in the protocol folder had to be created, where this is the main entry to the protocol and is executed by the node as the new amendment/consensus protocol. This is the only part that has to be implemented in OCaml and has to be compiled to work with the node.

To do so, a ``TEZOS\_PROTOCOL'' file must be included in this module, that is used to specify the version of the environment that the protocol is to be compiled against, which in this case the version 6 was the one used, the hash of the protocol folder and the set of modules implemented in the folder. 

There's the possibility to implement a newer environment for the protocol to accommodate the protocol, yet, for this experiment, this wasn't necessary. 

Like said previously, the environment is an interface that provides a set of OCaml modules that the protocol can use, and the interface used to interact with the Tezos Shell.

The most important functions and types in that had to be implemented for environment v6 were:

\begin{itemize}
    % TODO: , mudar isto para código, no nome das coisas
    \item block\_header\_data: This is the protocol-specific part of the header.
    \item block\_header: This is the combination of the protocol header and the shell header.
    \item operation\_data: This is the protocol-specific part of an operation/transaction.
    \item operation: This is the combination of the protocol operation part and the shell operation part.
    \item validation\_state: This is the state of the validation of a block or operation, passed between functions.
    \item init: This function is executed to prepare the chain to start executing the new protocol.
    \item begin\_application: This function is used when validating a block received from the network.
    \item begin\_partial\_application: This function is used when the shell receives a block more than one level ahead of the current head. 
    \item begin\_construction: This function is used by the shell when instructed to build a block and for validating operations as they are gossiped on the network.
    \item apply\_operation: This function is called after begin\_application or begin\_construction and before finalize\_block for each operation in the block or in the mempool, respectively. It validates the operation and updates the intermediary state accordingly.
    \item finalize\_block: This function represents the last step in a block validation sequence.
\end{itemize}

In summary, the TEZOS\_PROTOCOL file is used to specify the protocol environment to be used by the Tezos node, and the environment provides a set of functions and types that the protocol must implement to be able to operate within the Tezos network.

\subsection*{Implementation of the module ``lib\_protocol''}

The implementation of a new protocol in the Tezos codebase is a significant challenge.
In this particular case, the goal was not only to implement the new protocol, but to also learn how existing protocols are structured and implemented within the Tezos codebase. The approach was to study and follow closely the implementation of existing protocols, rather than taking an easier approach that may also have resulted in a working protocol but would not have offered the same level of learning.

The protocol that was implemented includes a few functionalities, such as the idea of a transaction between two entities, where Tez, the currency used in Tezos, is transferred from one account to another. 
Another aspect of the protocol is the concept of mining and verifying the proof of work, like mentioned in a previous section about Proof of Work.
This is done by hashing the header whole header, both the shell part and the protocol part, this last one containing the nonce, and verifying that the hash value is lower than the target where this is one of the main mechanism behind proof of work.

The environment that the protocol is compiled against requires a definition of the protocol header, which contains three elements: ``target'', ``nonce'', and ``miner''.

The \textbf{target} is used for comparison with the target value that the node calculated it should be (the specifics of how this is done is explained in a later point).
The \textbf{nonce} is used to achieve a header hash with a value lower than the target, which is a key part of the Proof of Work mechanism.
Finally, the \textbf{miner} field stores the address of the miner who found the nonce/hash, which is used to reward the miner. These elements combined make up the protocol-specific part of the header.


\subsection*{Implementation of Information Representation and Storage Logic}

The implementation of the protocol in Tezos involves defining and representing the various types of information that are necessary for the protocol to function. This includes things like accounts, constants, headers, time, target, currency (Tez), and operations.

The information stored in the protocol is abstracted, meaning that the protocol itself is stateless. The context in which the protocol is applied maps to the actual storage of the blockchain, but this is hidden from the protocol. The protocol only sees the storage as a generic key-value store and the functions that access this are defined in the "Raw\_context.ml" file (in reality, they are accessed as a Tree, not just as a generic key-value). These functions allow the protocol to read, write, and update the storage, but the actual details of how this is done are not relevant to this document.


The Tezos Protocol is implemented through a series of files that represent various components and functionalities. In this section, we will take a closer look at the information stored and the logic behind each component in the protocol.

The protocol contains representations/definitions and operations of the following types/information, which are stored in files that end with "repr.ml". The Tezos documentation calls this the ``representation layer'' of the protocol.

\begin{itemize}

    \item Accounts/Manager: This represents the concept of an account in the protocol. An account is just a key, which is used to fetch information from the storage. It is a Public Key Hash (of Ed25519, Secp256k1, P256).

    \item Constants/Parameters: This file contains information that is constant throughout the execution of the protocol. Some constants can be defined when activating the protocol. The constants include:

        \begin{itemize}
            \item block\_time: Defines the time between blocks.
            \item initial\_target: Defines the first target value.
            \item difficulty\_adjust\_epoch\_size: Defines the number of blocks to wait to then readjust the target.
            \item halving\_epoch\_size: Defines the number of blocks to wait to then halve the mining reward.
            \item reward\_multiplier: The initial reward, which then gets halved.
            \item Header: Defines the Protocol header, as mentioned before.
            \item block\_time: Defines the time between blocks.
        \end{itemize}

    \item Time: Defines the type of time used throughout the protocol.

    \item Target: Defines how the target should be, in this case, it's a 256-bit number.

    \item Tez: Defines the type and operations for the currency.

    \item Operations: Representations and functionalities of the available operations. These include Transactions (between two entities) and Reveal (maps a Public Key to a Public Key Hash, like the ones used in the account repr).

\end{itemize}

Some of these representations where designed by following the existing protocols implemented in the Tezos codebase. These representations ensure that the creation, conversion, encoding and decoding and other functionalities are done correctly by the protocol.

The logic behind storing information in the Tezos Protocol is an integral part of its functionality. The protocol uses the blockchain as its storage mechanism. 
The files that are responsible for storing information have names ending with "-storage.ml", and they store the following key pieces of information:
\begin{itemize}
    \item Accounts: It stores information about each account in the protocol. It maps the account's key to the account's balance and other important information and how defines how this information should be stored, retrieved and other checks.

    \item Target: It stores the current target, which is used for comparing the value of the header hash in the proof of work process.

    \item Epoch Time: It stores the timestamp of the latest difficulty adjust epoch, which is used to determine when it's time to adjust the target value. This is an important part of maintaining the security and integrity of the network, as it allows the network to dynamically adjust the difficulty of mining to ensure a stable rate of block production.
\end{itemize}

By storing these types of information in the blockchain, the Tezos Protocol provides a secure and transparent method of tracking the state of the network, which is crucial for the proper functioning of its operations. The protocol also provides a well-defined structure for the information it stores, which ensures consistency and maintainability of the code.

\subsection*{Implementation of the main entry points to the protocol}

The implementation of the main functions of the protocol plays a crucial role in ensuring the correct execution of the protocol. These functions are designed to take the necessary steps to apply the information and verify the validity of the information being applied. The functions presented in the previous section, such as "begin\_application", "begin\_partial\_application", "begin\_construction", "apply\_operation", "finalize\_block", and "init", are all a part of the main functions of the protocol.

These functions perform various checks and updates to the context, which is an immutable representation of the state of the protocol. The context contains information such as the accounts, the target, and the epoch time, among others. The functions return some form of validation state and updated context, if applicable.

\begin{itemize}

    \item begin\_application and begin\_partial\_application:
        \begin{itemize}
            \item These functions prepare the current raw context, which contains information such as the context and more. The concept of raw context will be explained later.
            \item They perform checks, such as verifying if the current target is the correct one and if the header has a valid hash, that is, if the hash of the whole blockheader has a value lower than the target.
            \item They update information such as the target for the next epoch, if an epoch has elapsed.
        \end{itemize}
    \item begin\_construction:

        \begin{itemize}
            \item Does the same preparation as begin\_application and begin\_partial\_application.
            \item Performs the same checks, but also rewards the miner if the block is valid.
        \end{itemize}

    \item apply\_operation:
        \begin{itemize}
            \item Verifies if the operation contains a valid signature.
            \item Verifies if the operation has a positive result, for example, if the account has enough funds to transfer to another account.
            \item If the operation is positive, it executes the operation on the current context.
        \end{itemize}

        \item finalize\_block:
        \begin{itemize}
            \item Commits the change to the shell.
            \item Returns a receipt or result log that exposes what happened with the application of either a block or operation.
        \end{itemize}
    \item init:
        \begin{itemize}
            \item Initializes everything needed to start the protocol, such as the constants and storage.
            \item Also creates the first blocked of the protocol to be appended to the chain.
        \end{itemize}
    \end{itemize}

The extensive logic involved in these functions will not be presented in this paper, but it is important to understand the role they play in ensuring the correct execution of the protocol. The protocol is stateless, meaning that the context is immutable and can only be updated in a controlled manner through the application of the functions.


\subsection*{Implementation of Alpha and Raw Contexts}

The implementation of the protocol relies heavily on two core abstractions: \textbf{Alpha\_context} and \textbf{Raw\_context}. These two modules play crucial roles in ensuring the separation of concerns and allowing the protocol to be implemented over a generic key-value store.

Alpha\_context, defined in the "alpha\_context.ml" module, serves as the consensus view of the context. This module enforces the separation between mapping the abstract state of the ledger to the concrete structure of the key-value store, and implementing the protocol over the state. The Alpha\_context defines a type 't' that represents the abstracted state of the ledger and can only be manipulated through the use of selected manipulations, which preserve the well-typed aspect and internal consistency invariants of the state.

The abstracted state is read from the disk during the validation of a block and is updated by high-level operations that preserve consistency. Finally, the low-level state is extracted to be committed to disk. This abstraction provides a well-separated structure in the code, with the code below Alpha\_context handling the ledger’s state storage, and the code on top of it implementing the protocol algorithm using plain OCaml values.

Raw\_context, defined in the "raw\_context.ml" module, is the information view used by Alpha\_context. It serves as the raw, storage, or non-abstract view of what is actually done in the background by the consensus/protocol. Raw\_context provides the abstraction used by Alpha\_context to access the raw part of the protocol, that is, the information that is actually stored and the representation layer.

In conclusion, the implementation of the protocol relies heavily on Alpha\_context and Raw\_context to enforce separation of concerns and to abstract away the underlying implementation details, allowing the protocol to be implemented over a generic key-value store in a readable and well-separated manner.

\subsection*{Features that weren't implemented}
In this project, we have focused on the implementation of consensus protocol in the Tezos node. However, there are a few features that have not been included in this study but could be considered as future work.

One of the missing features is the support for smart contracts. This was not the main focus of the experiment, but smart contracts can be added on top of the protocol to provide more functionality. Currently, the protocol only supports peer-to-peer value transactions, but with the addition of smart contracts, more complex operations can be performed.

Another missing feature is the lack of upgradability in the protocol. This was intentional as this experiment was focused on a standalone Proof of Work consensus algorithm, and there is no previous protocol nor will there ever be a next protocol that is an amendment to this one. Upgradability could be added to the protocol in the future to provide more flexible and dynamic updates, but once again, for this kind of experiment, doing so would be useless.

It should be noted that the absence of these features was not due to technical limitations, but rather a deliberate decision to focus on the implementation of a protocol in the Tezos.

\subsection*{Tools developed to interact with the protocol}

In this study, tools related to the protocol were also implemented to enable interaction with the network. These tools could be developed independently from the Tezos project, since it's possible to communicate with the tezos node through JSON RPC. However, implementing these tools using OCaml and Tezos modules has the added benefit of being automatically integrated with the Tezos node client. Upon activation of the protocol, the client would automatically adapt to accommodate it, enabling commands that are specific to the protocol, such as "Transfer" and "Reveal."

One of the tools implemented was the client, which can be found in the "lib\_client" folder. The client is mostly used to access information on the network, such as an account's balance or the current target. It serves as a better user interface for accessing the information through JSON RPC services. The client also has the capability of injecting blocks and operations into the network.

Another tool that was implemented is the baker, which can be found in the "lib\_baker" folder. The baker uses functionalities implemented in both the client module and the "Shell Services" module, this last one being already part of the Tezos codebase.
The baker performs the task of baking (how it's called in the whole Tezos project), or mining, a block by taking multiple operations, preapplying the block in the protocol (to check if the block header is valid), and then performing the work to find the nonce that makes the block's hash lower than the target, similar to what a miner would do in a proof-of-work blockchain. Once the block is mined, it can be pushed to the chain.\\


In conclusion, this experiment proved to be a valuable exercise as it allowed us to gain a deeper understanding of how consensus protocols work and how they can be implemented within a blockchain network. The implementation of the protocol allowed us to test its capabilities, limitations and to identify areas for improvement. Furthermore, it demonstrated the versatility and flexibility of the Tezos network in accommodating custom consensus protocols, making it an ideal platform for experimentation and innovation.

\section{Future Tasks}

The research plan outlined below highlights the tasks that we intend to undertake in this thesis project. However, it should be noted that the direction of the project may evolve and change as we progress, leading to potential modifications to the plan. Nonetheless, this serves as a starting point for our research journey.

\textbf{Task 1}: Implementation of the Proof of Work (experiment) protocol in Tezos. The goal of this task is to implement a Proof of Work protocol in the Tezos blockchain, which is a well-known platform for experimenting with new consensus algorithms.

\textbf{Task 2}: Testing of the Proof of Work protocol. In this task, we will test the Proof of Work protocol that was implemented in the previous task, to ensure that it is functioning correctly.

\textbf{Task 3}: Development of a generic framework for adding new consensus algorithms. The aim of this task is to make it easier and more straightforward to add new consensus algorithms to the Tezos platform. This will be achieved by developing a generic framework that can be easily adapted to support new protocols.

\textbf{Task 4}: Creation of a platform for testing consensus algorithms. In this task, we will create a platform that will allow us to test various consensus algorithms, including the Proof of Work protocol that was implemented in task 1. This platform will be used to compare the performance and efficiency of different protocols.

\textbf{Task 5}: Integration of a Domain-Specific Language (DSL) for consensus algorithms. The goal of this task is to make it easier to integrate new consensus algorithms into the Tezos platform, by using a DSL that can be adapted to support different protocols. The knowledge gained from the development of the generic framework for adding new consensus algorithms in task 3 will be used as input for this task.

\textbf{Task 6}: Development of additional tools for blockchain consensus algorithms. In this task, we will continue to build on the knowledge gained from the previous tasks, and develop additional tools that can be used to improve the performance and efficiency of blockchain consensus algorithms.

\textbf{Task 7}: Writing of the thesis. The thesis will present the research results and provide conclusions based on the work performed in the previous tasks. The writing of the thesis will be a final step, but will be done concurrently with the other tasks.

In addition to these tasks, we may consider incorporating additional features or making improvements to existing features, depending on the results of our research and the progress of our work. The aim is to continue advancing the state of the art in the field of blockchain consensus algorithms, and to provide a comprehensive understanding of the various protocols and tools that are available for improving the performance and efficiency of blockchain networks.

