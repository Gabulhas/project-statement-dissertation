\chapter{Problem Statement, Experiments and Work Plan}

\section{Introduction}
MUDAR:
The Problem Statement, Proposed Solution, Experiments, and Work Plan chapter is a crucial component of the thesis as it outlines the research problem, a possible solution to it, an experiment to further sustain that solution, and the plan for executing the research. In this chapter, we aim to present a clear and concise description of the problem we are solving, why it is important, and why the proposed solution is a valid one. The chapter will also include a detailed description of the experiments that will be conducted to validate the solution, as well as the work plan that outlines the tasks and the timeline for executing the research.

\section{The Problem}
The development of blockchain technology has led to the creation of a large number of consensus protocols, each with its unique characteristics, from the selection of the leader, to the type of network, data structure, block structure, time between each block, and multiple ways to have a chain, to name just a few. Newer and innovative protocols are designed ``everyday'' with totally different characteristics compared to previous protocols. There's a lot of ``fine tunning'' to be done, even when comparing protocols of the same type. These small changes are enough to start a totally different network. For example, just increasing the original block size of Bitcoin was enough to start a new network, Bitcoin Cash \cite{bitcoincash}.
\label{bitcoincashexample}

This abundance of choices has made it difficult for individuals or organizations looking to start their own blockchain to determine which protocol is the best fit for their needs, as there's a lot to account for.

Furthermore, there is no easy way to test and compare these consensus protocols in a live, non-simulated environment.
There is no blockchain node software that allow for a seamless swap of the consensus algorithm, making it challenging to properly evaluate the performance of each protocol. This presents a major obstacle for researchers and developers looking to experiment with and improve upon existing consensus protocols.

Additionally, the process of designing and coding a new consensus algorithm is also challenging and requires a significant amount of expertise and resources. There is currently no platform that allows for the easy design, swap, and testing of new consensus algorithms.

All these problems contribute to the lack of standardization in the field of consensus protocols, a lot of attributes and characteristic to take in, making it difficult for individuals and organizations to adopt blockchain technology effectively.
It is crucial to find a solution that allows for the easy and efficient comparison and testing of consensus protocols, as well as the design and implementation of new protocols. This will pave the way for the further development and widespread adoption of blockchain technology.


\section{Proposed Solution}
The proposed solution aims to tackle the challenges mentioned in the previous section by providing a method of easily swapping consensus algorithms, so they can then be tested. To achieve this, the solution leverages the advantages of a pre-existing blockchain that is suitable for both swapping and testing.

By using an already built blockchain, the focus of this solution is solely on the consensus protocol/layer, the crucial aspect of any blockchain. The need for implementing other parts of the blockchain node such as the Peer-to-Peer layer, client, validation layer, storage of the blockchain state, etc., is eliminated. This benefits the solution by making use of already tested and industrial-grade blockchain software, freeing up time and resources to concentrate solely on the consensus.

One of the ideas for implementing this solution is to use the Tezos blockchain. The reason for using Tezos is its inherent adaptability, which allows for the removal of its current consensus algorithm and the implementation of a new one. This approach provides a layer of adaptability to the Tezos blockchain, making it suitable for testing and swapping various consensus algorithms with ease.

As of the writing of this document, there's no knowledge of projects or documents that describe a similar solution to this one proposed. There exists the concept of consensus testing when talking about a blockchain node implementation in specific, that is, for example, tools to test the Ethereum network, or the Tezos network, but there are no tools to test specifically the consensus part of this projects and compare them.


In the next section, an experiment will be demonstrated to showcase the feasibility and practicality of our objective. This first study is designed to validate our ideas and provide a foundation for future development. While we did not develop a full-fledged platform/toolset, this experiment serves as a crucial first step in demonstrating that there's a potential for consensus algorithms to be easily swapped, tested, and compared in a live environment.


In conclusion, the solution proposes to overcome the challenges mentioned in the previous section by providing a method of testing and easily swapping consensus algorithms. By leveraging the advantages of a pre-existing blockchain and its adaptability, the focus remains solely on the consensus algorithm, the crucial aspect of any blockchain, allowing for efficient and effective testing and comparison of different consensus algorithms.




\section{Conclusion}
MUDAR:

In conclusion, this project has explored the crucial concepts of in the field of blockchain technology. Through a review of the state of the art, it was revealed that there are several popular consensus algorithms used in blockchain networks, each with its own strengths and weaknesses. The project then identified the challenges in the field of consensus protocols, including the lack of standardization, difficulties in comparison, testing and the fact that there's a multitude of different consensus algorithms available in the field of blockchain technology,
making it difficult for individuals and organizations to adopt blockchain effectively.

The proposed solution was to overcome these challenges by providing a method of testing and easily swapping consensus algorithms in a pre-existing well-tested blockchain network.
The implementation of this solution was tested and proved to be feasible by the execution of the experiment done, demonstrating the versatility and flexibility of the Tezos network in accommodating custom consensus protocols.

The document also outlined the main contributions, goals, and objectives for future work.
The next phase of the project will focus on the completion and testing of the Proof of Work (experiment) protocol implemented in Tezos, development of a generic framework for adding new consensus algorithms, creation of a platform for testing them, integration with the Lupin DSL, and writing of the thesis to present research results and conclusions on the work performed. The aim is to advance the state of the art in blockchain consensus algorithms and provide a comprehensive understanding of available protocols and tools to improve block\-chain network performance and efficiency.





