\chapter*{Resumo Alargado}
Esta dissertação debruça-se sobre os desafios intrincados e as soluções inovadoras associadas à implementação e teste de algoritmos de consenso em redes blockchain. O trabalho é centrado na plataforma blockchain Tezos, selecionada pela sua notável adaptabilidade e versatilidade. O objetivo primordial é contribuir para o avanço da tecnologia blockchain, criando ferramentas e métodos que permitam não só a implementação, mas também o teste rigoroso e a análise comparativa de diferentes algoritmos de consenso em ambientes de rede em tempo real.

O problema que esta dissertação procura resolver é duplo: primeiro, a falta de ferramentas que permitam a implementação ágil de algoritmos de consenso; segundo, a ausência de meios para testar esses algoritmos em condições de rede ao vivo. Estas lacunas tornam difícil para os investigadores e profissionais da indústria avaliar a eficácia e as limitações de diferentes algoritmos de consenso.

O documento está organizado em várias secções críticas. A introdução fornece uma visão geral do problema e da solução proposta, sublinhando as lacunas e desafios existentes. Uma revisão aprofundada do estado da arte em algoritmos de consenso segue-se, abordando uma variedade de algoritmos e as suas respectivas eficiências, limitações e aplicações.

Um estudo de caso preliminar é detalhado, focando na implementação de um algoritmo de Prova de Trabalho (PoW) na plataforma Tezos. Este estudo não só serve como uma prova de conceito, mas também oferece insights profundos sobre os mecanismos subjacentes dos algoritmos de consenso e os desafios de integrá-los em nós de blockchain existentes.

A contribuição principal desta dissertação reside no desenvolvimento de duas ferramentas específicas. A primeira, denominada 'Protocol Bootstrapper', é uma ferramenta de desenvolvimento que simplifica o processo de implementação de novos algoritmos de consenso. A segunda, conhecida como 'LiveTester', é uma ferramenta de teste em tempo real que permite aos utilizadores executar testes rigorosos em algoritmos de consenso em condições de rede ao vivo.

Por último, a dissertação apresenta um cenário prático, onde ambas as ferramentas são aplicadas na implementação e teste de um algoritmo de Prova de Autoridade (PoA). Os resultados deste teste são comparados com os do algoritmo PoW, fornecendo dados empíricos que destacam as eficiências e limitações de ambos os algoritmos.

Em resumo, esta dissertação oferece um quadro abrangente e robusto para o desenvolvimento, implementação e teste de algoritmos de consenso, apoiado por dados empíricos recolhidos na plataforma Tezos. Este trabalho tem implicações significativas para a comunidade académica e para os stakeholders da indústria, estabelecendo um novo padrão para futuras investigações e desenvolvimentos neste campo em rápido crescimento.
